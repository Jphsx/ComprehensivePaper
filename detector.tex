\subsection{Accelerator Description}
\label{ilc}


The search for new physics drives center of mass energies higher and higher. The current most powerful operating collider is the Large hadron collider, LHC at cern with center of mass energies of 13-14TeV. Since the dominating QCD background and busy environment that is created from many parton collision creates a significant challenge discoering new physics, the next frontier in high energy physics is naturally a linear accelrator(linac) through electron and positron collisions. This is because electron an positron are light particles they can not be accelerated to as high of energy as hadrons but are amenable to precision measurments because of the clean environment. The shape of the collider is also important because synchotron radiation is more with lighter particles but only if the particle changes direction.  The last major electron positron collider was LEP which reached center of mass energies of around 200 GeV which still provides the dominating electroweak precision measurments that are good today.  The ILC is next proposed future collider which would harness center of mass energies from 200GeV up to a possible 1 TeV upgrade.  The proposed design was originally to start at 500 GeV center of mass energy along a 30km linac. Prohibitive costs have pushed the starting center of mass energy to 250 GeV  with a 20 km linac with possible 500 GeV and 1 TeV upgrades as well as luminosity upgrades.  The starting luminosity planned to be achieved is 1.35 · 1034 cm−2s−1, leading to a integrated luminosity of  2 ab−1 after a decade. The acclerator will have an electron beam that 
polarised to $80\%$, and and positron source which will deliver a beam with a $30 \%$ positron polarisation.\cite{currdetector} A more detailed description of the accelrator designs can be found the Technical Design Report(TDR) \cite{TDR}.

\subsection{Detector Description}
\label{ild}

There are two proposed detectors at the ILC which would serve the same interaction point on a push-pull mechanism. Where one detector will take data while the other is under maintenance. This allows for continuous collection of data, the opportunity for complimentary detector designs,  offering the competition between detector experiments, and cross-checks between experimental results. All with the benefit of lower overall cost since there is only a single interaction point(IP). The two proposed detectors are the International Large Detector(ILD) and the Silicon Detector(SiD). Both detectors are shell components, outward from the interactions point is a vertex detector, tracker used to identify charged particles, a soleniod which produces the magnetic field bending incident charge particles. an electromagnetic calormiter which measures light particles and photons, a dense hadronic calorimter to stop and contain the showers from  heavier particles. and a external muon layer which detects muons. The forward regions have a collection of calormiters designed to capture beam particles scattered at small angles. Both detectors optimize reconstruction of particles by the use of the Particle Flow Algorithm(PFA). PFA is method that that combines algorhitms and a highly granular calorimiter to fully resolve individual particles and their energy deposits \cite{pfa}.
ILD approach to PFA optimization is by making the detector large, thus physically separating the particle more making reconstruction easier. SiD approach is towards cost efficiency with a smaller detector and strong magnetic field to achieve a similar performance. 
 The major difference between the two detector are the tracking mechanisms.  The ILD is planned to use a gaseous central tracker  with Time projection chamber. which provides a nearly continuous path information for tracks by providing up to 224 hits per track. SiD is planned to use a silicon tracking system similar to the LHC. The design demands for both detectors are as follows: at least 4$\mu$m spatial resolution in the vertex detector , momentum resolution $\Delta 1/p 2\time 10^{-5} \, \text{GeV}^-1$, a jet energy resolution $3\%$. and hermicity specifically to capture and conserve momentum from particles in the forward region in order to benefit analysis driven by missing energy.

\subsection{Software}
\label{ilcsoft}

The software ecosystem for ILC is contained under ILCSOFT \cite{ilcsoft} which is comprised of reconstruction tools that rely on the event data model LCIO. FUll simulation samples generated are based on detector descriptions in DD4HEP \cite{dd4hep} and physics samples centrally produced with Whizard \cite{ whizard}


