The results show a promising start to potential electro weak precision measurements for the ILC with the statistical error on the mass of $\Delta M_W\text{(stat.)}= 2.4$ MeV. This measurement is competitive with the current PDG full measurement of the W mass  of $\Delta M_W=12$ MeV.   Comparing with another study of mass measurements performed at $\sqrt{s} = 250$ GeV, the potentially achievable statistical error is $\Delta M_W\text{(stat.)} = 1.6$ MeV for a Gaussian mass resolution $\sigma_W= 4$ GeV \cite{graham}. The estimated uncertainties at ILC250 provide a glass ceiling for the measurement of the W-mass in ILC500 being $\Delta M_W =3.7$ MeV. The mass measurement at ILC500 is more challenging becuase the W pair overlapping with pile-up is not an issue at lower of center of mass eneriges, so, 3.7 MeV may be unobtainable at higher $\sqrt{s}$. Both measurements, however, are dominated by the systematic uncertainties from the effective jet energy scale which is a challenging demand.  The statsitical error on the cross-section also shows the utitilty of semileptonic WW as a method to precisely measure the beam polarization at the interaction point. This offers an important alternative for  correctly measuring processes that are sensitive to beam polarization and assists in quantifying of beam depolarization from collisions.

The lepton identfication and charge assignment performance is exceptional with a overall correct charge assignment with $98.8\%$ over all three channels. This has the biggest impact on the measurment of charged triple gauge couplings(TGC) that rely on the identification of the $W^-$. The lepton identification itself, still has room for improvement in ways such as (1) an MVA type approach with TauFinder and (2) reoptimization of the parameters over a mixed polarization beam-scenario which involves both LR and RL events. Currently the optimzation was performed on a purely LR configuration, this helicity combination has a distinct advantage of well isolated leptons, whereas in RL a well separated lepton is not guaranteed. The optimization of TauFinder parameters led to every isolation cone choosing 150 mrad,which is indicative of not achieving a global minimum. However, allowing the isolation to grow wider could mean overtuning the lepton identification to the topologies specific to LR. 

The pileup mitigation is a mostly unexplored avenue of reconstuction in ILC500 as most processes are produced centrally or are not simulated with pile-up. The techniques developed to remove pile-up are optimized for W mass measurement, but, can be easily adapted to general usage in any type of process where the standard approaches for pileup removal are inadequate. 

The event selection can also be improved. Additional cuts were explored such as the hadronic recoil against the leptonic W, the leptonic W mass, or maximum track multiplicity. The leptonic W mass cut best motivates the categorization approach of WW-like and not WW-like types of events, but, this cut, and others mentioned do not improved the overall efficiency times purity of the analysis.  Specifically, the leptonic W mass can be improved and applied to event selection by using more sophisticated calculation for the neutrino momemtum by taking into account potential ISR in the z-direction. This ISR compensation combined with a kinematic fit with constraints on the energy, momentum, and equal W masses would significantly improve the measured leptonic W mass and enhance the performance of the event selection. 
 
Some additional detector benchmarks can immediately follow the results of this analysis, one would be the quality of separation between prompt muons and tau muons to evaluate the performance of the vertex detector. Another study that should be done is to obtain the analysis efficiency as a function of the polar angle of the lepton testing the performance of the foward calorimiters.  Overall the semi leptonic analysis offers keen insights to analysis perfromed in an electron positron collider with $\sqrt{s} = 500$ GeV. The statistical errors on cross-section and mass are the first step but an important and tractable step in electro weak precision measurement at the ILC.  The semileptonic channel still offers significantly more imporant physics in terms of TGC and polarization measurements in addtion to expanding and improving the analysis presented here. 